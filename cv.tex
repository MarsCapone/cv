\documentclass{tccv}
\usepackage[english]{babel}
\usepackage[T1]{fontenc}
\usepackage{multicol}
\usepackage{fancyhdr}

\begin{document}

\twocolumn[

\part{Samson P. Danziger}

\begin{multicols}{2}
    \personal
        {samsondanziger.com}
        {31 Great James Street\newline London, WC1N 3HB}
        {+44 (0)7903 516311}
        {samson@samsondanziger.com}
        {github.com/MarsCapone}

    \section{About}
        I am a keen programmer interested in virtualisation, reading Computer Science at the University of Southampton. I am very persistent and reliable in the tasks I decide to do - if I say I will do something, I will definitely do it. I am currently working at Citrix, and designing a web game based on Mornington Crescent, from the BBC Radio 4 classic, I'm Sorry I Haven't a Clue.

\end{multicols}

\hline
\bigskip

\begin{multicols}{2}
    \section{Education}

        \begin{yearlist}

        \item{2017 -- Exp. 2019}
             {MEng Computer Science \newline w/ Industrial Studies}
             {University of Southampton}

        {\item[Awarded 2:1]{2014 -- 2017}
             {BSc Computer Science}
             {University of Southampton}}

        \end{yearlist}

    \section{Software skills}

        \begin{factlist}

        \item{Advanced}
             {Python, Java, Linux, Git}

        \item{Intermediate}
             {Ocaml, Scheme, Bash, \LaTeX, Windows}

        \item{Basic}
             {C, JavaScript, Node, HTML5, SQL}

        \end{factlist}

\end{multicols}

\hline
\bigskip
]

\section{Experience \& Projects}

    \begin{eventlist}
        \item{Jun. 2017 -- Exp. Jun. 2018}
             {Citrix}
             {Software Engineer}

        I am currently interning as a Software Engineer at Citrix, focused on XenServer. This is the lowest level of the system just above the hardware. I am a member of two teams, one maintains the the kernel and drivers. The other involves producing, and verifying XenServer builds. I am finding it exciting and interesting so far, particularly being able to use my skills in functional programming and furthering my understanding of low level operating system architecture.

        \item{Feb. 2016 -- May. 2016}
             {COMP2211 Coursework}
             {Software Engineering Group Project}

        I was the team leader in a group project to create a demo marking application for roadside driver assessments at Heathrow airport. The application was written in Java, and we used agile methods such as scrums, and sprint sessions to great success. I learnt not to assume that everyone else had the same knowledge as me, despite being on the same course.

        \item{Mar. 2016}
             {COMP2212 Coursework}
             {Splat Programming Language}

        Created a functional programming language, written in OCaml, as part of a group coursework. This was my first moderate sized project using a functional language. Following on from this success, I am working on an extension to Splat. This project cemented my knowledge of how evaluation, type checking and recursion worked.\newline\newline

        \item{Oct. 2013 -- Jul. 2014}
             {Valere Capital Partners}
             {IT Support}

        Installed and maintained computer equipment and software. Worked on upgrading the office network connection, and setting up VoIP for the office. I learnt to be systematic in fault-finding, and experienced working under pressure with users who were keen to have their problems solved quickly.


        \item{Jan. 2014 -- Jul. 2014}
             {Dragon Hall Youth Centre}
             {3D Printing Coordinator}

        Partially built and maintained 8 3D printers. Was involved in getting young people (ages 8 - 16) interested in, and able to use 3D printers.\\
        Over the course of various demonstrations and fund-raising events, I became much more confidant at presenting my ideas, and with public speaking.

        \item{Jan. 2014 -- Jun. 2014}
             {Clerkenwell Primary School}
             {Code Club Volunteer}

        I taught 15 children ages 10 to 11 how to make games using the Scratch programming language. Later we also moved onto the basics of Python. I learnt to be flexible when dealing with groups of mixed ability.

        \item{Oct. 2012 -- Jan. 2013}
             {Run by University of Southampton}
             {National Cipher Challenge}

        I compiled and led my team, the aptly named Winning Combination, to first place. Involved solving 8 pairs of increasingly difficult ciphers, with the final cipher (a modified trifid) taking 44 hours and 20 minutes to solve. We won \textsterling 1000, of which we donated 20\% to Code Club. As a result of this, I realised that even problems which seem insurmountable, can eventually be solved with time and persistence.

    \end{eventlist}


%\vfill

\section{Extracurricular}

\begin{eventlist}

    \item{2014 -- 2017}
         {University of Southampton}
         {Student Robotics}

    I was strongly involved in Student Robotics, a charity involved in getting 6th formers to build and program autonomous robots. I was involved in creating a Django based scheduler for competition roles. I also participated as a volunteer at the 2 day competition for a number of years, and have mentored three teams. I was voted in to be the secretary of the Southampton branch, making me responsible for the general management of the society and liaising with the Student's Union.

    \item{2014 -- 2017}
         {University of Southampton}
         {Toastrack Bus}

    I am (casually) the webmaster of the Toastrack Bus society, a small group of students who manage and try to maintain a 1920s bus. This society allowed me to develop my hands-on attitude - a welcome break from the theoretical side of my course - although I have now taken a step back from the main runnign of the society. More information at \href{http://toastrackbus.org}{toastrackbus.org}.

    \item{2014 -- 2017}
         {University of Southampton}
         {Code Dojo}

    I was a regular member of the Southampton Code Dojo, a monthly meet-up of tech individuals, coding some challenge in just over an hour. So far I have written a Mondrian-style art generator, a story generator, a web based music make, and a project to find and separate clusters on a graph. My usual language is Python, but I have also dabbled in Javascript and Scheme. As I am no longver living in Southampton, I hope to find a new Code Dojo, or similar event closer to my current address.

    \item{2016}
         {University of Southampton}
         {Ballroom Dancing}

    In early 2016 I learnt ballroom dancing, and participated in a competition after only a month. I did not come last, which I consider to be a great success. This was also a personal challenge, I as usually find it difficult to present myself to large groups of people.\newline\newline

    \item{}
         {}
         {Sports \& Hobbies}

    I enjoy less traditional activities such as circus skills. I am fairly competent at unicycling, juggling and diabolo. When I have time I go long distance walking, skiing or sailing. I have walked the length of Hadrian's wall and the North Norfolk Coastal Path. I spent the summer of 2013 becoming a ski instructor, and I have also participated in a 6 week tour of the Peloponnese and the Tall Ships races.

\end{eventlist}

%\vfill

\section{Awards \& Qualifications}

\begin{factlist}
\item{Driving}{Full Clean Driving License}
\item{Sailing}{RYA Level 4, Competent Crew}
\item{Ski Instructor}{BASI Level 1 \& Level 2 Teaching}
\item{First Aid}{12 Hour First Aid Certificate}
\item{Art}{Art Sculpture Prize}
\item{Other}{ECDL Certificate}
\end{factlist}

%\vfill

\section{References}

\begin{eventlist}

\item{Supervisor}
     {Citrix Systems UK}
     {Jacus De Beer}

\href{mailto:jacus.debeer@citrix.com}{jacus.debeer@citrix.com}
\newline
+44 12 2322 5911

\bigskip


\item{University Tutor}
     {University of Southampton}
     {Markus Brede}

\href{mailto:mb8@ecs.soton.ac.uk}{mb8@ecs.soton.ac.uk}
\newline
+44 23 8059 3703

\end{eventlist}




\end{document}
